\documentclass{beamer}

\mode<presentation>
{\usetheme{boxes}}

\usepackage{array}
\usepackage{times}
\usepackage{graphicx}
\usepackage{hyperref}
\usepackage{listings}
\usepackage{relsize}
\usepackage{ragged2e}
\usepackage[T1]{fontenc}

\lstdefinestyle{customc}{
  belowcaptionskip=1\baselineskip,
  breaklines=true,
  frame=L,
  xleftmargin=\parindent,
  language=C,
  showstringspaces=false,
  basicstyle=\footnotesize\ttfamily,
  keywordstyle=\bfseries\color{green!40!black},
  commentstyle=\itshape\color{purple!40!black},
  identifierstyle=\color{blue},
  stringstyle=\color{red},
}
\lstdefinestyle{custombash}{
  belowcaptionskip=1\baselineskip,
  breaklines=true,
  frame=L,
  xleftmargin=\parindent,
  language=bash,
  basicstyle=\footnotesize\ttfamily,
  showstringspaces=false,
  commentstyle=\itshape\color{purple!40!black},
  keywordstyle=\itshape\color{green!40!black},
  identifierstyle=\color{blue},
  stringstyle=\color{orange},
}

\usebackgroundtemplate
{
  \hbox to \paperwidth{\hfil\includegraphics[width=4in,
      height=\paperheight]{wildcat_transparent.jpg}\hfil}
}

\title{PHYS 105 Lecture 11: Simulating a Bouncing Ball}
\author{Tom McClintock \\
	Dept. of Physics\\
	University of Arizona
}
\date{\today}

\begin{document}

\begin{frame}
  \titlepage
\end{frame}

\begin{frame}
  \frametitle{Last time}
  \begin{itemize}
    \item Gaussians
    \item Dynamic Displays
  \end{itemize}
\end{frame}

\begin{frame}
  \frametitle{This time}
  \begin{itemize}
    \item Simulating a Bouncing Ball
  \end{itemize}
\end{frame}

\begin{frame}
  \frametitle{Gravity}
  From day one in physics you learned learned about the force of gravity:
  \begin{equation*}
    \vec{F} = m\vec{g}.
  \end{equation*}
  This class, we are going to investigate this force numerically. 
  Specifically, we are going to simulate a ball released from rest some height
  above a hard surface.
\end{frame}

\begin{frame}
  \frametitle{Disclaimer!}
  Yes, this problem is trivially solved using energy, and
  completely reduces to the kinematic equations, but the 
  numerical methods here are general, and apply to problems with
  any complicated force or initial configuration.
\end{frame}

\begin{frame}
  \frametitle{Equations of Motion I}
  On the screen, we will plot the height of the ball, $y$,
  and presume that in the $x$ direction it is always at $x=0$.\\
  Since the ball is falling, the height is dependent on the time, $y(t)$.
  Therefore, in a small chunk of time called a ``timestep'', $\Delta t$,
  the height will change:
  \begin{equation*}
    \Delta y = v\Delta t.
  \end{equation*}
\end{frame}

\begin{frame}
  \frametitle{Equations of Motion II}
  Similarly, since the force of gravity is pulling on the ball,
  the velocity, $v$, is also changing:
  \begin{equation*}
    \Delta v = a\Delta t.
  \end{equation*}
  What is the acceleration $a$? It is given by the force
  \begin{equation*}
    a = \frac{F}{m}
  \end{equation*}
  where
  \begin{equation*}
    F = -mg
  \end{equation*}
  and the $-$ means the force points down, or in the $-y$ direction.
\end{frame}

\begin{frame}
  \frametitle{Equations of Motion III}
  Together, these equations constitute the ``equations of motion'' of the
  system, and demonstrate the steps that we need to take to find the height.
  \begin{itemize}
    \item Calculate $F$
    \item Calculate $a = \frac{F}{m}$.
    \item Calculate the change in velocity $\Delta v = a\Delta t$.
    \item Calculate the change in position $\Delta y = v\Delta t$.
    \item Update the velocity and positions:
      \begin{itemize}
      \item $y(t+\Delta t) = y(t) + \Delta y$
      \item $v(t+\Delta t) = v(t) + \Delta v$
      \end{itemize}
    \item Repeat.
  \end{itemize}
\end{frame}

\begin{frame}
  \frametitle{Equations}
  Here are all of the equations, you will need:
  \begin{eqnarray*}
    &F = -mg  \\
    &a = \frac{F}{m}\\
    &\Delta v = a\Delta t  \\
    &\Delta y = v\Delta t\\
    &y(t+\Delta t) = y(t) + \Delta y \\
    &v(t+\Delta t) = v(t) + \Delta v\\
    &t = t + \Delta t
  \end{eqnarray*}
  For constants we will use $m = 1kg$, and for initial
  conditions we will have $y(0) = 5 m$ and $v(0)=0m/s$.
  Now we are ready to program this.
\end{frame}

\begin{frame}[fragile,allowframebreaks]
  \frametitle{bounce.c}
  \lstinputlisting[style=customc]{bounce.c}
\end{frame}

\begin{frame}[fragile]
  \frametitle{Wut?}
  Ok, so the ball fell through the floor. Why was that?\\
  It was because we never said what to do when the ball reached
  the floor. This is an example of a \textbf{boundary condition} in
  our simulation. In general, bounday conditions control what happens
  at the extremum of your code, and are needed to make it work properly.
  To fix our code, add the following lines at the end of the for loop
  just after the delpoint\_plot():
  \begin{lstlisting}[style=customc]
    if(y < 0){
      v = -v;
    }
  \end{lstlisting}
  This will turn the ball around.
\end{frame}

\begin{frame}
  \frametitle{In class assignment}
  Analytically, dealing with friction is very difficult. 
  Numerically this is no problem though.
  When calculating the force, all you have to do is add on a 
  new term that acts in the opposite direction
  of the velocity. This looks like:
  \begin{equation}
    \vec{F}_f = -bv^2\hat{v}.
  \end{equation}
  Thus, if the velocity is down, the force points up, 
  and if the velocity is up, it points down. Implement this
  frictional force, and find a value for $b$ so that the ball 
  bounces to only half of its original height on its first bounce.
\end{frame}

\begin{frame}
  \frametitle{Next time}
  \begin{itemize}
  \item Projectile Motion
  \end{itemize}
\end{frame}

\begin{frame}
  \frametitle{HW 6 - photon off of a mirror due in one week}
  Create a program that bounces a photon (dot) off of a mirror. Have
  the photon come in from the top left corner, hit the mirror in the center,
  and leave through the top right corner.
\end{frame}

\end{document}