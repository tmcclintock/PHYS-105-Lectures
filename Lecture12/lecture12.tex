\documentclass{beamer}

\mode<presentation>
{\usetheme{boxes}}

\usepackage{array}
\usepackage{times}
\usepackage{graphicx}
\usepackage{hyperref}
\usepackage{listings}
\usepackage{relsize}
\usepackage{ragged2e}
\usepackage[T1]{fontenc}

\lstdefinestyle{customc}{
  belowcaptionskip=1\baselineskip,
  breaklines=true,
  frame=L,
  xleftmargin=\parindent,
  language=C,
  showstringspaces=false,
  basicstyle=\footnotesize\ttfamily,
  keywordstyle=\bfseries\color{green!40!black},
  commentstyle=\itshape\color{purple!40!black},
  identifierstyle=\color{blue},
  stringstyle=\color{red},
}
\lstdefinestyle{custombash}{
  belowcaptionskip=1\baselineskip,
  breaklines=true,
  frame=L,
  xleftmargin=\parindent,
  language=bash,
  basicstyle=\footnotesize\ttfamily,
  showstringspaces=false,
  commentstyle=\itshape\color{purple!40!black},
  keywordstyle=\itshape\color{green!40!black},
  identifierstyle=\color{blue},
  stringstyle=\color{orange},
}

\usebackgroundtemplate{
  \hbox to \paperwidth{\hfil\includegraphics[width=4in,
      height=\paperheight]{wildcat_transparent.jpg}\hfil}
}

\title{PHYS 105 Lecture 12: Projectile Motion}
\author{Tom McClintock \\
	Dept. of Physics\\
	University of Arizona
}
\date{\today}

\begin{document}

\begin{frame}
  \titlepage
\end{frame}

\begin{frame}
  \frametitle{Last time}
  \begin{itemize}
    \item Simulating a Bouncing Ball
  \end{itemize}
\end{frame}

\begin{frame}
  \frametitle{This time}
  \begin{itemize}
    \item Projectile Motion
  \end{itemize}
\end{frame}

\begin{frame}
  \frametitle{2D Projectile Motion}
  Projectile motion was a staple for your introductory mechanics class.
  Hopefully you recall that it is actually pretty simple to go from
  one dimension to two. Just write all of your kinetic equations
  twice: once for the $x$ direction and once for the $y$ direction.\\
  The same is true for our simulation.
\end{frame}

\begin{frame}
  \frametitle{Equations of Motion}
  Here are our equations of motion now:
  \begin{eqnarray*}
    F_x = 0 && F_y = -mg\\
    \vec{a} &=& \frac{\vec{F}}{m}\\
    \Delta\vec{v} &=& \vec{a}\Delta t\\
    \Delta\vec{r} &=& \vec{v}\Delta t\\
    \vec{r}(t+\Delta t) &=& \vec{r}(t) + \Delta\vec{r}\\
    \vec{v}(t+\Delta t) &=& \vec{v}(t) + \Delta\vec{v}
  \end{eqnarray*}
  The vector equations just mean that you write it twice: once for
  the $x$ direction and once for the $y$ direction.
\end{frame}

\begin{frame}[fragile,allowframebreaks]
  \frametitle{projectile.c}
  \lstinputlisting[style=customc]{projectile.c}
\end{frame}

\begin{frame}
  \frametitle{In class assignment}
  Now, including drag in our calculation is a little trickier, since it
  can in general have a component in the $x$ or $y$ direction. These will
  look like the following:
  \begin{eqnarray*}
    &f_x = -bv^2\hat{v}\cdot\hat{x} = -bv^2\left(\frac{v_X}{v}\right)\\
    &f_y = -bv^2\hat{v}\cdot\hat{y} = -bv^2\left(\frac{v_y}{v}\right)
  \end{eqnarray*}
  Implement these equations by including them in the forces.\\
  Also, make your projectile bounce when it hits the ground (so, reverse $v_y$
  when $y<0$).
\end{frame}

\begin{frame}
  \frametitle{Next time}
  \begin{itemize}
  \item Orbital Motion
  \item Final Projects
  \end{itemize}
\end{frame}

\begin{frame}
  \frametitle{Last HW - Harmonic Motion due in three weeks}
  Simulate a ball hanging from the ceiling by a spring, 
  and include air resistance.
  The force for this would look like the following:
  \begin{equation*}
    F_y = -mg - bv^2\frac{v_y}{v} - ky
  \end{equation*}
  where $\vec{F}$ always points in the $y$ direction. Here,
  $y$ is the displacement from the origin.
  Simulate this with initial conditions $y(0)=+1.0$ and $v(0)=-1.0$, 
  with constants of $b=0.1$ and $k = 20.0$.
  You can quickly write this homework by modifying your projectile motion
  program.
\end{frame}

\end{document}
