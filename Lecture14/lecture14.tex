\documentclass{beamer}

\mode<presentation>
{\usetheme{boxes}}

\usepackage{array}
\usepackage{times}
\usepackage{graphicx}
\usepackage{hyperref}
\usepackage{listings}
\usepackage{relsize}
\usepackage{ragged2e}
\usepackage[T1]{fontenc}

\lstdefinestyle{customc}{
  belowcaptionskip=1\baselineskip,
  breaklines=true,
  frame=L,
  xleftmargin=\parindent,
  language=C,
  showstringspaces=false,
  basicstyle=\footnotesize\ttfamily,
  keywordstyle=\bfseries\color{green!40!black},
  commentstyle=\itshape\color{purple!40!black},
  identifierstyle=\color{blue},
  stringstyle=\color{red},
}
\lstdefinestyle{custombash}{
  belowcaptionskip=1\baselineskip,
  breaklines=true,
  frame=L,
  xleftmargin=\parindent,
  language=bash,
  basicstyle=\footnotesize\ttfamily,
  showstringspaces=false,
  commentstyle=\itshape\color{purple!40!black},
  keywordstyle=\itshape\color{green!40!black},
  identifierstyle=\color{blue},
  stringstyle=\color{orange},
}

\usebackgroundtemplate{
  \hbox to \paperwidth{\hfil\includegraphics[width=4in,
      height=\paperheight]{wildcat_transparent.jpg}\hfil}
}

\title{PHYS 105 Lecture 14: Large Programs and Makefiles}
\author{Tom McClintock \\
	Dept. of Physics\\
	University of Arizona
}
\date{\today}

\begin{document}

\begin{frame}
  \titlepage
\end{frame}

\begin{frame}
  \frametitle{Last time}
  \begin{itemize}
    \item Orbital Motion
  \end{itemize}
\end{frame}

\begin{frame}
  \frametitle{This time}
  \begin{itemize}
    \item Large Programs
    \item Makefiles
  \end{itemize}
\end{frame}


\begin{frame}
  \frametitle{Large Programs}
  As you may have noticed, our programs are growing ever larger.
  This becomes a problem if you want an individual file to be readable.
  In general, it is good coding philosophy to break a large file into
  smaller pieces that all do a self-contained operation. This is the
  philosophy we will take in breaking apart some of our larger programs.
\end{frame}

\begin{frame}
  \frametitle{Breaking apart gaussian.c}
  Today, we will break apart the gaussian.c program that we wrote
  way back when. Feel free to copy/paste into a new file, but I will
  be writing the code from scratch.\\
  Our strategy will be to put the gaussian function, previously written
  just below the main(), into its own file.
\end{frame}

\begin{frame}
  \frametitle{Header files}
  The way you link c-files together is through ``header'' files. These
  files contain the prototypes of the functions found in an identically
  named c-file. For this program, we will create a header 
  called ``function.h'' for the c-file called ``function.c''.
\end{frame}

\begin{frame}[fragile]
  \frametitle{function.h}
  This file, called ``function.h'' contains the prototype and library links
  used in the code with the same name but a ``.c'' suffix.
  \lstinputlisting[style=customc]{function.h}
\end{frame}

\begin{frame}[fragile]
  \frametitle{function.c}
  The function itself is in a c-file.
  \lstinputlisting[style=customc]{function.c}
\end{frame}

\begin{frame}[fragile]
  \frametitle{gauss\_main.c}
  The entire main is in its own file.
  \lstinputlisting[style=customc]{gauss_main.c}
\end{frame}

\begin{frame}
  \frametitle{Superglue: the Makefile}
  As you may have noticed, the compilation instructions are different.
  For programs with many c-files and headers, it is a pain to compile
  at the command line. For this reason, we make a Makefile. A Makefile
  is like a recipe: it contains the instructions on how to make
  our executable.
\end{frame}

\begin{frame}
  \frametitle{Makefile}
  \lstinputlisting[style=customc]{Makefile}
\end{frame}

\begin{frame}
  \frametitle{Compiling and running}
  To compile a program with a Makefile all you have to do is type ``make''.
  To run you just type the name of the executable.
\end{frame}

\begin{frame}
  \frametitle{In class assignment}
  Work on your final project. Have a good Thanksgiving!
\end{frame}

\begin{frame}
  \frametitle{Next time}
  \begin{itemize}
  \item Python programs
  \end{itemize}
  Work on your final project!
\end{frame}

\begin{frame}
  \frametitle{HW 7 - Harmonic Motion due in one week}
  Simulate a ball hanging from the ceiling by a spring, 
  and include air resistance.
  The force for this would look like the following:
  \begin{equation*}
    \vec{F} = m\vec{g} - bv^2\hat{v} - k\vec{d}
  \end{equation*}
  where $\vec{F}$ always points in the $y$ direction. Here,
  gravity points down, $\hat{v}$ points in the direction
  of the velocity, and $\vec{d}$ is the displacement from
  equilibrium of the spring (i.e. distance from the starting position).
  Simulate this with initial conditions $y(0)=0$ and $v(0)=0$.
  You can quickly write this homework by modifying your dropping ball program.
\end{frame}

\end{document}
