\documentclass{beamer}

\mode<presentation>
{\usetheme{boxes}}

\usepackage{array}
\usepackage{times}
\usepackage{graphicx}
\usepackage{hyperref}
\usepackage{listings}
\usepackage{relsize}
\usepackage{ragged2e}
\usepackage[T1]{fontenc}

\lstdefinestyle{customc}{
  belowcaptionskip=1\baselineskip,
  breaklines=true,
  frame=L,
  xleftmargin=\parindent,
  language=C,
  showstringspaces=false,
  basicstyle=\footnotesize\ttfamily,
  keywordstyle=\bfseries\color{green!40!black},
  commentstyle=\itshape\color{purple!40!black},
  identifierstyle=\color{blue},
  stringstyle=\color{red},
}
\lstdefinestyle{custombash}{
  belowcaptionskip=1\baselineskip,
  breaklines=true,
  frame=L,
  xleftmargin=\parindent,
  language=bash,
  basicstyle=\footnotesize\ttfamily,
  showstringspaces=false,
  commentstyle=\itshape\color{purple!40!black},
  keywordstyle=\itshape\color{green!40!black},
  identifierstyle=\color{blue},
  stringstyle=\color{orange},
}
\lstdefinestyle{custompython}{
  belowcaptionskip=1\baselineskip,
  breaklines=true,
  frame=L,
  xleftmargin=\parindent,
  language=bash,
  basicstyle=\footnotesize\ttfamily,
  showstringspaces=false,
  commentstyle=\itshape\color{purple!40!black},
  keywordstyle=\itshape\color{green!40!black},
  identifierstyle=\color{blue},
  stringstyle=\color{orange},
}

\usebackgroundtemplate{
  \hbox to \paperwidth{\hfil\includegraphics[width=4in,
      height=\paperheight]{wildcat_transparent.jpg}\hfil}
}

\title{PHYS 105 Lecture 15: Other languages and Python}
\author{Tom McClintock \\
	Dept. of Physics\\
	University of Arizona
}
\date{\today}

\begin{document}

\begin{frame}
  \titlepage
\end{frame}

\begin{frame}
  \frametitle{Last time}
  \begin{itemize}
  \item Large Programs
  \item Makefiles
  \end{itemize}
\end{frame}

\begin{frame}
  \frametitle{This time}
  \begin{itemize}
  \item Other programming languages
  \item Python
  \end{itemize}
\end{frame}

\begin{frame}
  \frametitle{Other programming languages}
  There are dozens and dozens of programming languages. You will never use
  more than a few of them. Some of the most popular are:
  \begin{itemize}
  \item C/C++
  \item C\#
  \item Python
  \item Java
  \item Matlab
  \item Ruby
  \item Fortran
  \end{itemize}
  Each language was designed with a different goal in mind. There is no
  one language that is absolutely better than any other. 
  \textit{The real skill in programming is knowing which language is the best to use for your project!}
\end{frame}

\begin{frame}
  \frametitle{A programmers' Swiss army knife}
  Roughly, these are what some of these languages are good at.
  \begin{itemize}
  \item C/C++/Fortran: intense number crunching, system optimization
  \item Java: cross-platform stability, graphical user interfaces
  \item Python: File I/O, data analysis, scripting
  \end{itemize}
  Today, we will learn how to use \textbf{Python}, a language which is
  overtaking many others in the fields of physics and astronomy as
  the most useful for data analysis.
\end{frame}

\begin{frame}
  \frametitle{Python}
  Python is a lot more ``relaxed'' than C. Python is weakly-typed, in other
  words the language figures out variable types for you. Also, the syntax
  is less stringent. There are no ; or curly braces \{\}.\\
  This class you will learn how to do some basic plotting using Python,
  and see how quickly you can create a histogram from your old 
  walker\_data.txt.
\end{frame}


\begin{frame}[fragile,allowframebreaks]
  \frametitle{example.py}
  \lstinputlisting[style=custompython]{example.py}
\end{frame}

\begin{frame}
  \frametitle{Running a python program}
  To run a python program you just type ``python myprogram.py'' and it 
  will run.\\
  \vspace{12pt}
  For the example program you should type ``python example.py''.
  \\
  \vspace{12pt}
  Next, let's see how simple it is to make a histogram in python. Let's
  use our old walker\_data.txt. You should copy it (with ``cp'') into the
  directory you are currently working in.
\end{frame}

\begin{frame}[fragile,allowframebreaks]
  \frametitle{pyhist.py}
  \lstinputlisting[style=custompython]{pyhist.py}
\end{frame}  

\begin{frame}
  \frametitle{In class assignment}
  Work on your final projects!\\ \textbf{They are due next week on December 9}.\\
  Remember to add lots of comments!
\end{frame}

\begin{frame}
  \frametitle{Next time}
  \begin{itemize}
  \item More python
  \end{itemize}
  Work on your final project!\\\textbf{They are due next week on December 9}.\\
  Also...
\end{frame}

\begin{frame}
  \frametitle{TCE Reports}
  These are necessary! Please complete a Teacher-Course Evaluation at:
  \href{http://tce.arizona.edu}{\nolinkurl{http://tce.arizona.edu}},
  and then click on ``Students: Rate your courses online'' on the right side
  of the page.
\end{frame}

\end{document}
