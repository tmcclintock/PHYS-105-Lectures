\documentclass{beamer}

\mode<presentation>
{\usetheme{boxes}}

\usepackage{times}
\usepackage{graphicx}
\usepackage{hyperref}
\usepackage{listings}
\usepackage{relsize}
\usepackage[T1]{fontenc}

\lstdefinestyle{customc}{
  belowcaptionskip=1\baselineskip,
  breaklines=true,
  frame=L,
  xleftmargin=\parindent,
  language=C,
  showstringspaces=false,
  basicstyle=\footnotesize\ttfamily,
  keywordstyle=\bfseries\color{green!40!black},
  commentstyle=\itshape\color{purple!40!black},
  identifierstyle=\color{blue},
  stringstyle=\color{red},
}
\lstdefinestyle{custombash}{
  belowcaptionskip=1\baselineskip,
  breaklines=true,
  frame=L,
  xleftmargin=\parindent,
  language=bash,
  basicstyle=\footnotesize\ttfamily,
  showstringspaces=false,
  commentstyle=\itshape\color{purple!40!black},
  keywordstyle=\itshape\color{green!40!black},
  identifierstyle=\color{blue},
  stringstyle=\color{orange},
}

\usebackgroundtemplate
{
  \hbox to \paperwidth{\hfil\includegraphics[width=4in,
      height=\paperheight]{wildcat_transparent.jpg}\hfil}
}

\title{PHYS 105 Lecture 3: Errors, If/Else, Blocks \& Scoping, Math Library}
\author{Tom McClintock \\
	Dept. of Physics\\
	University of Arizona
}
\date{\today}

\begin{document}

\begin{frame}
  \titlepage
\end{frame}

\begin{frame}
  \frametitle{Last time}
  \begin{itemize}
    \item Finding your HW grades
    \item More terminal commands
    \item Variables and basic math
    \item I/O from the user
  \end{itemize}
\end{frame}

\begin{frame}
  \frametitle{This time}
  \begin{itemize}
    \item More Terminal Commands
    \item Errors!
    \item If/Else statements
    \item Blocks \& Scoping
    \item Math library
  \end{itemize}
  \vspace{12pt}
  Remember to make a new directory for today's lecture.\\
  Use the \textbf{history} command or look at the lecture 2 slides
  to remind yourself how to do this.
\end{frame}

\begin{frame}
  \frametitle{More Terminal Commands}
  \begin{itemize}
    \item \textbf{rm \textit{filename}} - this stands for ``remove''. 
      It permanently deletes the file. You can NEVER GET IT BACK.
    \item \textbf{mv \textit{source} \textit{destination}} - this stands
      for ``move'' and serves two puproses. First it can rename files 
      (e.x. ``mv oldname newname''). Second, it can move files to new
      locations such as a directory (e.x. ``mv hello.c Lecture1/'').
    \item \textbf{cp \textit{source} \textit{destination}} - this stands
      for ``copy''. You can make copies of files 
      (e.x. ``cp hello.c hello\_copy.c'').
  \end{itemize}
\end{frame}

\begin{frame}[fragile]
  \frametitle{Compiler Errors}
  The vast majority of errors you will encounter will be compiler errors. That is,
  the compiler (gcc) complains that you messed something up when it tries to compile
  your code.\\
  Let's look again at hello.c.
  \lstinputlisting[style=customc]{hello.c}
\end{frame}

\begin{frame}[fragile]
  \frametitle{Compiler Errors}
  \lstinputlisting[style=customc]{hello.c}
  If you noticed, this time I left off the necessary semicolon (;)
  on the line with printf(). This will cause a compiler error.
\end{frame}

\begin{frame}[fragile]
  \frametitle{Compiler Errors}
  If I try to compile
  \begin{lstlisting}[style=custombash]
    gcc hello.c -o hello.exe
  \end{lstlisting}
  I recieve the message
  \begin{lstlisting}[style=custombash]
    hello.c: In function `main':
    hello.c:11: error: expected `;' before '}' token
  \end{lstlisting}
  Thus, the compiler told me there is an error within \textbf{main()} on line 11.\\
  It also says that it expected a `;' before it went on to reach the end of main `\}'.\\
  Read error messages when they appear and you will debug much faster.
\end{frame}

\begin{frame}[fragile]
  \frametitle{If Statements}
  There are often times where your program wants to ``branch'' into two or more paths.\\
  This is where \textbf{if} statements come into play.
  \begin{lstlisting}[style=customc]
    ... //outside the if statement
    if(condition){
      ... //inside the if statement
    }
    ... //outside the if statement
  \end{lstlisting}
  The code that is inside the if statement will \textbf{only} occur 
  if the condition is evaluated as true.
\end{frame}

\begin{frame}[fragile]
  \frametitle{Else Statements}
  An \textbf{if} statement by itself isn't enough to branch more than once.\\
  This is where \textbf{else} comes into play.
  \begin{lstlisting}[style=customc]
    ... 
    if(condition1){
      ... //code block 1
    }else if(condition2){
      ... //code block 2
    }else{
      ... //code block 3
    }
    ...
  \end{lstlisting}
  In this example, we can see that if the condition1 is true, then code block 1 will execute.\\
  If the condition1 is false and condition2 is true then code block 2 will execute.\\
  If condition1 and condition2 are both false then code block 3 will execute.
\end{frame}

\begin{frame}[fragile]
  \frametitle{Conditions}
  Conditions are \textit{usually} mathematical statements.\\
  Here are some examples using a variables called \textbf{x} and \textbf{y}.\\
  \begin{lstlisting}[style=customc]
    x < y  //x is less than y
    x > y  //x is greater than y
    x == y //x is equal to y, Note the `double equals' sign used for comparison
    x != y //x is not equal to y
    x <= y //x is less than or equal to y
    x >= y //x is greater than or equal to y
  \end{lstlisting}
  Note that either \textbf{x} or \textbf{y} could 
  have been numbers rather than variables.
\end{frame}

\begin{frame}[fragile]
  \frametitle{Combining conditions}
  You can combine conditions to achieve ``and'' as well as ``or'' statements:
  \begin{lstlisting}[style=customc]
    x > 1 && x < 3 //x is greater than 1 AND x is less than 3
    y < -1 || y == 9 //y is less than -1 OR y is equal to 9
  \end{lstlisting}
  You can also invert statements (i.e. add ``not''):
  \begin{lstlisting}[style=customc]
    !(i == 7) //i is not equal to 7
    !(x < y) //x is not less than y
  \end{lstlisting}
\end{frame}

\begin{frame}[fragile,allowframebreaks]
  \frametitle{Program with if/else}
  An example program where the conditions are based on user input.
  %\vspace{12pt}  \vspace{12pt}
  \lstinputlisting[style=customc]{ifelse.c}
  Notice that testing if the number is positive doesn't require a condition.\\
  It is assumed that if the number isn't negative (condition1), 
  and isn't equal to 0 (condition2)
  then it must be positive.\\
  It is often helpful to put comments near if/else statements as I have done.
\end{frame}

\begin{frame}[fragile]
  \frametitle{Blocks}
  Earlier your saw the use of the word \textbf{block}. 
  This concept is very simple.\\
  A block is any piece of code contained within curly braces \{\}.
  \lstinputlisting[style=customc]{blocks.c}
  We can see above that blocks are 
  often \textbf{nested} inside of each other.\\
  The comments indicate the start and end of the two blocks in this program.\\
  Don't bother writing this program.
\end{frame}

\begin{frame}[fragile,allowframebreaks]
  \frametitle{Scoping}
  \textbf{Blocks} are important because \textit{variables only 
    exist within the blocks they were declared in and 
    within any nested blocks}.\\
  Consider the following code snippet:
  \begin{lstlisting}[style=customc]
    if (condition1){
      int i = 0; //i is declared here
      ... //i exists and can be used here
    }else{
      ... //i doesn't exist here
    }
    ... //i doesn't exist here
  \end{lstlisting}
  Because \textbf{i} was declared within the block following 
  the \textbf{if} statement it doesn't
  exist outside that block.\\
  This is very convenient, because if variables existed forever after 
  they were declared then it is likely
  that we could override a critical variable within the computer!!!
\end{frame}

\begin{frame}[fragile]
  \frametitle{Math Library}
  You might have noticed this magical line at the top of all of your programs
  \begin{lstlisting}[style=customc]
    #include <stdio.h>
  \end{lstlisting}
  This includes the input/output library, 
  which allows us to use printf() and scanf().\\
\end{frame}

\begin{frame}[fragile,allowframebreaks]
  \frametitle{Math Library}
  Now, you will learn about the most useful library to physicists, 
  the math library.\\
  You include it in much the same way as stdio.h.
  \lstinputlisting[style=customc]{mathlibrary.c}
  \textbf{Critical note:} you must include ``-lm'' when you compile a program 
  that has the math library!
\end{frame}

\begin{frame}[fragile]
  \frametitle{In Class Assignment}
  The trig function $sin(x)$ will work no matter what the value of $x$ is.
  However, the inverse sine, a.k.a. $arcsin(x)$ or $asin(x)$, only works 
  if $-1\le x\le1 $. Write a program
  that takes a floating point number from the user, and checks to see if it is
  a valid input for $arcsin$. If it is, print out the value of
  $asin(x)$ from the math library. 
  If not, tell the user it isn't a valid input.
\end{frame}

\begin{frame}
  \frametitle{Next time}
  HW 2 is due!
  \begin{itemize}
    \item Binary
    \item More types
    \item Using sizeof()
    \item For loops
  \end{itemize}
\end{frame}

\begin{frame}
  \frametitle{HW 2 - due next week}
  Write a program that takes three floats from the user: a length, width, and height of a box.\\
  After taking in the three numbers, calculate the volume of the box and print it out.\\
  Remember to comment your code! The more the better!
\end{frame}


\end{document}