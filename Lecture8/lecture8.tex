\documentclass{beamer}

\mode<presentation>
{\usetheme{boxes}}

\usepackage{array}
\usepackage{times}
\usepackage{graphicx}
\usepackage{hyperref}
\usepackage{listings}
\usepackage{relsize}
\usepackage{ragged2e}
\usepackage[T1]{fontenc}

\lstdefinestyle{customc}{
  belowcaptionskip=1\baselineskip,
  breaklines=true,
  frame=L,
  xleftmargin=\parindent,
  language=C,
  showstringspaces=false,
  basicstyle=\footnotesize\ttfamily,
  keywordstyle=\bfseries\color{green!40!black},
  commentstyle=\itshape\color{purple!40!black},
  identifierstyle=\color{blue},
  stringstyle=\color{red},
}
\lstdefinestyle{custombash}{
  belowcaptionskip=1\baselineskip,
  breaklines=true,
  frame=L,
  xleftmargin=\parindent,
  language=bash,
  basicstyle=\footnotesize\ttfamily,
  showstringspaces=false,
  commentstyle=\itshape\color{purple!40!black},
  keywordstyle=\itshape\color{green!40!black},
  identifierstyle=\color{blue},
  stringstyle=\color{orange},
}

\usebackgroundtemplate
{
  \hbox to \paperwidth{\hfil\includegraphics[width=4in,
      height=\paperheight]{wildcat_transparent.jpg}\hfil}
}

\title{PHYS 105 Lecture 8: Graphics, Random Numbers and Random Walk}
\author{Tom McClintock \\
	Dept. of Physics\\
	University of Arizona
}
\date{\today}

\begin{document}

\begin{frame}
  \titlepage
\end{frame}

\begin{frame}
  \frametitle{Last time}
  \begin{itemize}
    \item Functions
    \item Pointers
  \end{itemize}
\end{frame}

\begin{frame}
  \frametitle{This time}
  \begin{itemize}
    \item Graphics
    \item Random numbers
    \item Random walk
  \end{itemize}
\end{frame}

\begin{frame}[fragile]
  \frametitle{PhilsPlot}
  The graphics package we will use is called PhilsPlot. It is designed by
  Professor Phil Pinto from the astronomy department.\\
  This class you will learn how to draw straight lines, and use this
  to plot a ``random walk'', which emulates a basic but important physical
  process.
\end{frame}

\begin{frame}[fragile]
  \frametitle{Including PhilsPlot}
  In all graphical programs you need to include the Philsplot library:
  \begin{lstlisting}[style=customc]
    #include ``philsplot.h''
  \end{lstlisting}
  In addition, we will use a \textbf{new compiler}. Instead of using 
  ``gcc'' from now on use ``pc'':
  \begin{lstlisting}[style=custombash]
    pc myprogram.c -o myprogram.exe
  \end{lstlisting}
  This stands for ``Philsplot compiler''.
\end{frame}

\begin{frame}[fragile,allowframebreaks]
  \frametitle{drawing.c}
  This program draws a line across the screen.
  \lstinputlisting[style=customc]{drawing.c}
\end{frame}

\begin{frame}
  \frametitle{Random numbers}
  Generating random numbers is an \textbf{essential} tool to
  science. We use random numbers to do statistics on data,
  perform difficult integrals, and do simulations.\\
  Today you will simulate a ``random walk''.\\
  Random numbers are generated from a random number generator (RNG).
  The RNG we will use is found in the \textit{stdlib.h} library.
\end{frame}

\begin{frame}[fragile]
  \frametitle{Random number basics}
  To call an RNG and get a random number, you use:
  \begin{lstlisting}[style=customc]
    int i;
    i = rand();
  \end{lstlisting}
  The function \textbf{rand()} returns a random number between 0 and
  the maximum allowed integer.\\
  To get a random number between, say 0 and 9, you use \textit{modulo}:
  \begin{lstlisting}[style=customc]
    int i;
    i = rand()%10;
  \end{lstlisting}
\end{frame}

\begin{frame}[fragile,allowframebreaks]
  \frametitle{randomnumbers.c}
  \lstinputlisting[style=customc]{randomnumbers.c}
\end{frame}

\begin{frame}
  \frametitle{Random walks}
  In physics, when particles move around in the ``real world''
  they are always taking a random walk. There are no billiard ball
  collisions at the microscopic level.\\
  A random walk simulator attempts to mimic this behavior.\\
  \vspace{12pt}
  We will start by considering the most basic random walk: 
  a particle takes $N$ steps, and each time can either step $+1$
  or step $-1$ (i.e. left or right).
\end{frame}

\begin{frame}
  \frametitle{Random walk algorithm}
  For the random walk our 
  approach will be the following:
  \begin{enumerate}
  \item Create a plot where the x-axis goes from 0 to the number of steps $N$ 
    and represents the number of steps taken so far $n$,
    and the y-axis goes from $\pm N$ and represents the 
    position of the particle $p$.
  \item Draw a random number $R$ that is either $-1$ or $+1$.
  \item If $R==-1$ step left $(p=p-1)$, otherwise step right $(p=p+1)$.
  \item Use drawto\_plot() to draw to the new location $(n,p)$
  \item Repeat this for $N$ steps.
  \end{enumerate}
\end{frame}

\begin{frame}[fragile,allowframebreaks]
  \frametitle{randomwalk.c}
  Let's take ten steps, so $N=10$.
  \lstinputlisting[style=customc]{randomwalk.c}
\end{frame}

\begin{frame}
  \frametitle{In class assignment}
  Today there are two assignments:
  \begin{enumerate}
  \item In \textbf{drawing.c} you drew two sides of a triangle. Complete the
    third side.
  \item Try modifying the random walk code to print out the current step
    number and position at each iteration. Also, alter the program so that
    the line changes color at each step.
  \end{enumerate}
\end{frame}

\begin{frame}
  \frametitle{Next time}
  \begin{itemize}
    \item File I/O
    \item Histograms
  \end{itemize}
\end{frame}

\begin{frame}
  \frametitle{No HW this week}
  There is no homework assignment this class.
\end{frame}

\end{document}