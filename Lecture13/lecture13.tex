\documentclass{beamer}

\mode<presentation>
{\usetheme{boxes}}

\usepackage{array}
\usepackage{times}
\usepackage{graphicx}
\usepackage{hyperref}
\usepackage{listings}
\usepackage{relsize}
\usepackage{ragged2e}
\usepackage[T1]{fontenc}

\lstdefinestyle{customc}{
  belowcaptionskip=1\baselineskip,
  breaklines=true,
  frame=L,
  xleftmargin=\parindent,
  language=C,
  showstringspaces=false,
  basicstyle=\footnotesize\ttfamily,
  keywordstyle=\bfseries\color{green!40!black},
  commentstyle=\itshape\color{purple!40!black},
  identifierstyle=\color{blue},
  stringstyle=\color{red},
}
\lstdefinestyle{custombash}{
  belowcaptionskip=1\baselineskip,
  breaklines=true,
  frame=L,
  xleftmargin=\parindent,
  language=bash,
  basicstyle=\footnotesize\ttfamily,
  showstringspaces=false,
  commentstyle=\itshape\color{purple!40!black},
  keywordstyle=\itshape\color{green!40!black},
  identifierstyle=\color{blue},
  stringstyle=\color{orange},
}

\usebackgroundtemplate{
  \hbox to \paperwidth{\hfil\includegraphics[width=4in,
      height=\paperheight]{wildcat_transparent.jpg}\hfil}
}

\title{PHYS 105 Lecture 13: Orbital Motion}
\author{Tom McClintock \\
	Dept. of Physics\\
	University of Arizona
}
\date{\today}

\begin{document}

\begin{frame}
  \titlepage
\end{frame}

\begin{frame}
  \frametitle{Last time}
  \begin{itemize}
    \item Projectile Motion
  \end{itemize}
\end{frame}

\begin{frame}
  \frametitle{This time}
  \begin{itemize}
    \item Orbital Motion
  \end{itemize}
\end{frame}

\begin{frame}
  \frametitle{Orbital Motion}
  Many systems exhibit harmonic motion. Orbital systems
  are one such example, and here we will make a crude
  approximaton of the Earth around the Sun.
\end{frame}

\begin{frame}
  \frametitle{Freedom of Units}
  Since we are the ones writing these programs we have
  full control over what units to use. We could use meters,
  kilometers, or some other distance scale. Similarly, we could
  use grams, kilograms, or some other mass unit (or even
  imperial units... barf!). This is because the lengths
  in philsplot don't actually correspond to any physical length.\\
  For this reason, we will choose a special set of units in which
  $GM_{sun}=1$. This is called using \textit{normalizing} your variables.
\end{frame}

\begin{frame}
  \frametitle{Equations of Motion}
  Here are our equations of motion now:
  \begin{eqnarray*}
    F_x = -\frac{m}{r^2}\frac{x}{r} && F_y = -\frac{m}{r^2}\frac{y}{r}\\
    \vec{a} &=& \frac{\vec{F}}{m}\\
    \Delta\vec{v} &=& \vec{a}\Delta t\\
    \Delta\vec{r} &=& \vec{v}\Delta t\\
    \vec{r}(t+\Delta t) &=& \vec{r}(t) + \Delta\vec{r}\\
    \vec{v}(t+\Delta t) &=& \vec{v}(t) + \Delta\vec{v}
  \end{eqnarray*}
  Notice the similarities with how everything is written compared to the
  projectile motion! With this formulation, as long as you can write
  down the forces, then everything else is identical!!!
\end{frame}

\begin{frame}
  \frametitle{Initial conditions}
  The uniqueness of a system such as this is completely set by the
  initial conditions. For this, let's use:
  \begin{eqnarray*}
    x = 1 && y = 0\\
    v_x = 0 && v_y = 1\\
    m = 1
  \end{eqnarray*}
\end{frame}

\begin{frame}[fragile,allowframebreaks]
  \frametitle{orbital.c}
  \lstinputlisting[style=customc]{orbital.c}
\end{frame}

\begin{frame}[allowframebreaks]
  \frametitle{Projects}
  Projects are due by \textbf{December 9}, the last day of classes. 
  This is a list of suggested topics, 
  however if you have an idea for a project then feel free to pursue that.
  I ask that if you pick your own project that you meet with me so
  we can discuss if it.
  \begin{itemize}
  \item Graphically simulate two billiard balls colliding on a pool table.
    You should include the interactions with the walls and the 
    effect of friction.
  \item Make an animation of a multibody system, e.g. three or more gravitating
    planets.
  \item Direct a photon into a lens and have it refract at each surface. This
    would serve as a demonstration (or simulation) of the bending of light in
    an optical system. Or choose to direct the photon onto one or more curved
    mirrors.
  \item Direct a photon of a fixed wavelength onto a prism and have it refract 
    at each surface by an amount which depends on its wavelength in a 
    reasonable way.
  \item Direct a non-relativistic charged particle towards a fixed charged 
    particle and plot the path.
  \item Create Lissajous figures for one or more chaotic systems.
  \item Simulate a double pendulum, or a double spring system.
  \item Plot the height versus time of a rocket blasting off from the
    surface of the Earth. A major fraction of the initial mass of the
    rocket is its fuel. It runs out of fuel a short time after blastoff.
  \item Implement a Monte-Carlo program (i.e. using random numbers) to
    calculate $\pi$ to some high degree of accuracy.
  \item Compare trajectories of projectile motion including different
    kinds of friction, e.g. linear ($-a\vec{v}$) or quadratic ($-bv^2\hat{v}$).
  \item Implement a higher order integration routine, such as $4^{th}$ order 
    Runge-Kutta.
  \item Write a program that finds the most accurate step size for
    the trapezoid integration method for the integral 
    $\int_0^{20}xe^{-x}dx$. 
    This is called the Poisson distribution.
  \item Write a program that does linear regression.
  \item Write a program that uses a recursive function that does something,
    such as finds the $N^{th}$ number in the Fibonacci sequence or calculates
    a factorial ($N!$).
  \item Implement a sorting algorithm that sorts an array of random
    integers.
  \item Implement a data structure such as a linked list or tree, and 
    demonstrate an advantage over an array.
  \item Develop a program to draw a maze as well as its solution. Keep this
    simple, but be able to demonstrate how it finds the solution.
  \item Find the most cost effective way for a person to travel to a number
    of cities (8 or more), given that the person wants to visit each city
    only once and, at the end, return to the city of origin. You should
    specify how you define ``cost effective'', e.g. shortest total distance
    traveled.
  \end{itemize}
\end{frame}

\begin{frame}
  \frametitle{In class assignment}
  \begin{enumerate}
    \item Try messing with the initial conditions such as the position and
      initial velocity.
    \item A good check on how realistic a simulation does 
      is by checking conserved
      quantities. Have your program print out the potential, kinetic, and
      total energies
      every ten steps. Is the total energy conserved?
  \end{enumerate}
\end{frame}

\begin{frame}
  \frametitle{Next time}
  \begin{itemize}
  \item Large Programs
  \item Basic makefiles
  \end{itemize}
  Remember to choose a project soon!
\end{frame}

\begin{frame}
  \frametitle{HW 7 - Harmonic Motion due in two weeks}
  Simulate a ball hanging from the ceiling by a spring, 
  and include air resistance.
  The force for this would look like the following:
  \begin{equation*}
    F_y = -mg - bv^2\frac{v_y}{v} - ky
  \end{equation*}
  where $\vec{F}$ always points in the $y$ direction. Here,
  $y$ is the displacement from the origin.
  Simulate this with initial conditions $y(0)=+1.0$ and $v(0)=-1.0$, 
  with constants of $b=0.1$ and $k = 20.0$.
  You can quickly write this homework by modifying your projectile motion
  program.
\end{frame}

\end{document}
