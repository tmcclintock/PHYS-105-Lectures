\documentclass{beamer}

\mode<presentation>
{\usetheme{boxes}}

\usepackage{array}
\usepackage{times}
\usepackage{graphicx}
\usepackage{hyperref}
\usepackage{listings}
\usepackage{relsize}
\usepackage{ragged2e}
\usepackage[T1]{fontenc}

\lstdefinestyle{customc}{
  belowcaptionskip=1\baselineskip,
  breaklines=true,
  frame=L,
  xleftmargin=\parindent,
  language=C,
  showstringspaces=false,
  basicstyle=\footnotesize\ttfamily,
  keywordstyle=\bfseries\color{green!40!black},
  commentstyle=\itshape\color{purple!40!black},
  identifierstyle=\color{blue},
  stringstyle=\color{red},
}
\lstdefinestyle{custombash}{
  belowcaptionskip=1\baselineskip,
  breaklines=true,
  frame=L,
  xleftmargin=\parindent,
  language=bash,
  basicstyle=\footnotesize\ttfamily,
  showstringspaces=false,
  commentstyle=\itshape\color{purple!40!black},
  keywordstyle=\itshape\color{green!40!black},
  identifierstyle=\color{blue},
  stringstyle=\color{orange},
}
\lstdefinestyle{custompython}{
  belowcaptionskip=1\baselineskip,
  breaklines=true,
  frame=L,
  xleftmargin=\parindent,
  language=bash,
  basicstyle=\footnotesize\ttfamily,
  showstringspaces=false,
  commentstyle=\itshape\color{purple!40!black},
  keywordstyle=\itshape\color{green!40!black},
  identifierstyle=\color{blue},
  stringstyle=\color{orange},
}

\usebackgroundtemplate{
  \hbox to \paperwidth{\hfil\includegraphics[width=4in,
      height=\paperheight]{wildcat_transparent.jpg}\hfil}
}

\title{PHYS 105 Lecture 16: More Python}
\author{Tom McClintock \\
	Dept. of Physics\\
	University of Arizona
}
\date{\today}

\begin{document}

\begin{frame}
  \titlepage
\end{frame}

\begin{frame}
  \frametitle{Last time}
  \begin{itemize}
  \item Other programming languages
  \item Python
  \end{itemize}
\end{frame}

\begin{frame}
  \frametitle{This time}
  \begin{itemize}
  \item More Python
  \end{itemize}
\end{frame}

\begin{frame}
  \frametitle{Visualization}
  A theme in this class is visualization. How can you visualize some data?
  How can you simulate a process?\\
  Python has fantastic facilities for visualizing things. In today's class
  we will use a tool unavailable in philsplot, which is the gradient map.
  In addition, we will see a quick and cool example of a 3D plot.
\end{frame}

\begin{frame}
  \frametitle{Gradient map}
  A gradient map shows some kind of change over an area. The simple program
  that follows animates a gradient map. Pretend that it shows heat flowing
  from a heater, when in reality it is just a way to visualize a 2 dimensional
  array.
\end{frame}

\begin{frame}[fragile,allowframebreaks]
  \frametitle{animation.py}
  \lstinputlisting[style=custompython]{animation.py}
\end{frame}

\begin{frame}
  \frametitle{3D plots}
  In other languages, 3D plots are horrendously difficult to make.
  The only other language that I know that can easily generate 3D plots
  is MATLAB, but that language is proprietary (aka it costs \$).
\end{frame}
\begin{frame}
  \frametitle{Python 3D plots}
  In the following, you will visualize the following function in 3D:
  \begin{equation*}
    Z(x,y,t) = \left(1-\sqrt{x^2}\right)\cos(2\pi x+\phi(t)).
  \end{equation*}
  That is, the height of the function will be $Z$, while $x$ and $y$ are coordinates
  in a horizontal plane. $Z$ is also a function of time $t$. There is also
  a comment in the program that you can un-comment, in order to see the
  equation
  \begin{equation*}
    Z(x,y,t) = \left(1-\sqrt{x^2+y^2}\right)\cos(2\pi x+\phi(t)).
  \end{equation*}
  Both $x$ and $y$ range from $[-1,1]$.
\end{frame}

\begin{frame}[fragile,allowframebreaks]
  \frametitle{wire3d.py}
  \lstinputlisting[style=custompython]{wire3d.py}
\end{frame}

\begin{frame}
  \frametitle{In class assignment}
  Congrats on finishing the semester! Good luck in PHYS 305!\\
  Please complete a Teacher-Course Evaluation at:
  \href{http://tce.arizona.edu}{\nolinkurl{http://tce.arizona.edu}},
  and then click on ``Students: Rate your courses online'' on the right side
  of the page.
\end{frame}

\end{document}
