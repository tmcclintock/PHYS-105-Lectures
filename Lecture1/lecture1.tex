\documentclass{beamer}

\mode<presentation>
{\usetheme{boxes}}

\usepackage{times}
\usepackage{graphicx}
\usepackage{hyperref}
\usepackage{listings}
\usepackage{relsize}
\usepackage[T1]{fontenc}

\lstdefinestyle{customc}{
  belowcaptionskip=1\baselineskip,
  breaklines=true,
  frame=L,
  xleftmargin=\parindent,
  language=C,
  showstringspaces=false,
  basicstyle=\footnotesize\ttfamily,
  keywordstyle=\bfseries\color{green!40!black},
  commentstyle=\itshape\color{purple!40!black},
  identifierstyle=\color{blue},
  stringstyle=\color{red},
}
\lstdefinestyle{custombash}{
  belowcaptionskip=1\baselineskip,
  breaklines=true,
  frame=L,
  xleftmargin=\parindent,
  language=bash,
  basicstyle=\footnotesize\ttfamily,
  showstringspaces=false,
  commentstyle=\itshape\color{purple!40!black},
  keywordstyle=\itshape\color{green!40!black},
  identifierstyle=\color{blue},
  stringstyle=\color{orange},
}

\usebackgroundtemplate
{
  \hbox to \paperwidth{\hfil\includegraphics[width=4in,
      height=\paperheight]{wildcat_transparent.jpg}\hfil}
}

\title{PHYS 105: Introduction to Computational Physics Lecture 1}
\author{Tom McClintock \\
	Dept. of Physics\\
	University of Arizona
}
\date{\today}

\begin{document}

\begin{frame}
  \titlepage
\end{frame}

\begin{frame}
  \frametitle{Logging on to the machine}
  Log onto the windows machine in front of you using your D2L login name 
  and password.
\end{frame}

\begin{frame}
  \frametitle{Logging on to faraday}
  Log into the school server by double-clicking on the SSH Secure Shell Client
  %Insert the picture here
  \begin{itemize}
  \item Server name: faraday
  \item Account name: your D2L login name
  \item Password: uap\_6214275
  \end{itemize}
\end{frame}

\begin{frame}
  \frametitle{Welcome to PHYS105!}
  Credit: One Unit\\
  Room: PAS 272\\
  Co-requisites: PHYS141 or PHYS161.\\
  Lectures: Tuesday 2-4 pm {\it or} Wednesday 2-4 pm.\\
  \vspace{12pt}
  Instructor: Tom McClintock\\
  Office: PAS 439\\
  Office Hours: Mondays 11-12 and Thursdays 9-11\\
  Email: \href{mailto:tmcclintock@email.arizona.edu}{\nolinkurl{tmcclintock@email.arizona.edu}}\\
  \vspace{12pt}
  TA: Alex Thompson\\
  Email: \href{mailto:alejandrot@email.arizona.edu}{\nolinkurl{alejandrot@email.arizona.edu}}
\end{frame}

\begin{frame}
  \frametitle{Syllabus}
  In this class you will:
  \begin{itemize}
  \item Learn the C programming language
  \item Use the Linux operating system
  \item Apply programming to physics problems
  \item Learn basic graphical techniques
  \item Learn how to simulate physical systems
  \end{itemize}
\end{frame}

\begin{frame}
  \frametitle{Syllabus}
  \textbf{Homeworks:} Approximately eight assignments throughout the semester. 
  All parts of the homework including source files and accompanying files 
  (such as output) must be submitted through D2L. 
  All files must be titled using the following format:
  lastname\_assignment.extension. Example: McClintock\_HW1.txt or 
  Thompson\_Final\_Project.c.\\
  \vspace{12pt}
  \textbf{Project:} Each student will complete a final project that 
  is due during the last week of class. More details will be discussed 
  at a later date.\\
  \vspace{12pt}
  \textbf{Final Exam:} There is no final exam for this course.
\end{frame}

\begin{frame}
  \frametitle{Syllabus}
  \textbf{Late Work:} Homeworks and projects handed in beyond the due-date
  will be accepted only if arrangementes have been made in advance. If
  you feel a grade for a homework or project was determined incorrectly
  you must submit a re-grade request within one week of the time the
  assignment was returned to the class.
\end{frame}

\begin{frame}
  \frametitle{Syllabus}
  \textbf{Grading:} Grades will be based on the homeworks and the 
  final project. Homeworks will count for 75\% of the final grade
  and the project will count for 25\%.\\
  Letter grades will be determined from numerical grades as follows:\\
  \indent 90-100 A, 80-89 B, 70-79 C, 60-69 D, <60 E.\\
  \vspace{12pt}
  \textbf{Incompletes:} Will be given if a student has 
  \underline{satisfactorily} completed the majority of the work in the class 
  and has a valid reason, such as  medical, for not completing the 
  remainder of the course. Students must make arrangements with the 
  instructor in order to receive an incomplete.
\end{frame}

\begin{frame}
  \frametitle{Syllabus}
  \textbf{Absence Policy:} All holidays or special events observed 
  by organized religions will be honored for those students who show
  affiliation with that particular religion.\\
  Absences pro-approved by the UA Dean of Students will be honored.\\
  \vspace{12pt}
  \textbf{Classroom Behavior:} Students are expected to maintain a good 
  level of decorum in the classroom. Please turn off all pagers (lol)
  and cell phones during class.
\end{frame}

\begin{frame}
  \frametitle{Syllabus}
  \textbf{General:} Students with disabilities who require reasonable 
  accomodations to fully participate in course activities or meet
  course requirements are encouraged to register with the Disability
  Resource Center and to contact the instructor to discuss access issues.\\
  \vspace{12pt}
  Students are expected to follow the University code of academic integrity
  and the code of student conduct. Those codes can be found at\\
  \href{http://deanofstudents.arizona.edu/policiesandcodes}{\nolinkurl{deanofstudents.arizona.edu/policiesandcodes}}\\
  \vspace{12pt}
  Other than grade and absence policies, the information in this syllabus
  may be subject to change with reasonable advance notice.\\
  \vspace{12pt}
  A copy of this syllabus can be found in its own document on the D2L website.
\end{frame}

\begin{frame}[fragile]
  \frametitle{Opening programs}
  C programs (as well as programs in any other language) are written in 
  text editors. You can use any editor you like, but I encourage you to use
  emacs.\\
  In your terminal, write:
  \begin{lstlisting}[style=custombash]
    emacs hello.c
  \end{lstlisting}
  And hit enter.\\
  A new window will open, which will allow you to type up a program. An annoying text box
  will be in the bottom. There is a bit of text you can click on to make it go away.
\end{frame}

\begin{frame}[fragile]
  \frametitle{Your first program}
  \lstinputlisting[style=customc]{hello.c}
  %\vspace{12pt}
  Lines with a // are called \textbf{comments}. 
  They are \textbf{necessary}.\\
  You can write whatever you want in a comment. They are used to explain
  what is going on to someone reading your code.\\
  You will be graded on how well you comment your code.
\end{frame}

\begin{frame}[fragile]
  \frametitle{Your first program}
  \lstinputlisting[style=customc]{hello.c}
  All programs in the C language begin at the line ``main()'' and proceed
  downward from there.\\
  In this program, the words ``Hello World!'' are printed to the screen.\\
  \textbf{Note:} Any line inside of a set of \{\} must end in a semicolon (;).
\end{frame}

\begin{frame}[fragile]
  \frametitle{Running your first program}
  Save your program by hitting control-x, or clicking the floppy disk at the top,
  or use your mouse to go to file$\rightarrow$save current buffer.\\
  In the terminal, type these two lines.
  \begin{lstlisting}[style=custombash]
    gcc hello.c -o hello.exe
  \end{lstlisting}
  and then
  \begin{lstlisting}[style=custombash]
    hello.exe
  \end{lstlisting}
\end{frame}

\begin{frame}[fragile]
  \frametitle{Running your first program}
  What happened here?
  \begin{lstlisting}[style=custombash]
    gcc hello.c -o hello.exe
  \end{lstlisting}
  \begin{lstlisting}[style=custombash]
    hello.exe
  \end{lstlisting}
  The first line \textbf{compiles} your code. It turns the C program into something
  the computer can read (the language of computers; ones and zeros).\\
  The second line ran the \textbf{executable} file.
\end{frame}

\begin{frame}[fragile]
  \frametitle{Homework 1}
  Due before the start of the next class.\\
  Change the message printed to the screen to something else.\\
  Submit your hello.c program to D2L.\\
  In order to submit to D2L, you must open firefox \textbf{from the terminal}.
  This can be done by simply typing
  \begin{lstlisting}[style=custombash]
    firefox
  \end{lstlisting}
  and hitting enter. From there navigate to d2l.arizona.edu and submit on the
  class webpage.
\end{frame}

\begin{frame}
  \frametitle{Next time}
  \begin{itemize}
    \item More terminal commands
    \item Variables and basic math
    \item I/O from the user
  \end{itemize}
\end{frame}
\end{document}